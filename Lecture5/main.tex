\documentclass{article}[a4paper, 12pt]
\usepackage[left=1in,right=1in,top=1.25in,bottom=1.25in]{geometry}
\usepackage{ctex}
\usepackage{amsthm}
\usepackage{hyperref}
\usepackage{amsmath}
\usepackage{amssymb}
\usepackage{tikz}
\usepackage{pgfplots}
\usetikzlibrary{arrows.meta}
\pgfplotsset{compat=1.17}

\newtheoremstyle{mystyle}
  {1em} % Space above
  {1em} % Space below
  {} % Body font
  {} % Indent amount
  {\bfseries} % Theorem head font
  {.} % Punctuation after theorem head
  {.5em} % Space after theorem head
  {} % Theorem head spec (can be left empty, meaning ‘normal’)

\theoremstyle{mystyle}
\newtheorem{problem}{题}
\newtheorem*{remark}{注记}
\newenvironment{solution}{\begin{proof}[解]}{\end{proof}}

\begin{document}

\title{复分析第五次习题课}
\author{彭子鱼}
\date{2024 年 5 月 26 日\\上次更新: \today}

\maketitle

\section{作业}

\begin{problem}[5.1.2]
  求函数在给定域上的Laurent展开.
  \begin{itemize}
    \item[(1)] \(\frac{1}{z^2(z-1)}\), \(D=B(1,1)\backslash\{1\}\);
    \item[(3)] \(\text{Log}\left(\frac{z-1}{z-2}\right)\), \(D=B(\infty,2)\);
    \item[(5)] \(\frac{1}{(z-5)^n}\), \(n\ge0\), \(D=B(\infty,5)\).
  \end{itemize}
\end{problem}

\begin{solution}
  \begin{itemize}
    \item [(1)] 由于\(|z-1|<1\), \[\begin{aligned}
      \frac{1}{z^2(z-1)}&=\frac{1}{z-1}\cdot\frac{1}{(1+z-1)^2}\\
      &=\frac{1}{z-1}\cdot\sum_{n=0}^\infty (-1)^n(n+1)(z-1)^n\\
      &=\sum_{n=0}^\infty (-1)^n(n+1)(z-1)^{n-1}.
    \end{aligned}\]
    \item [(3)] 易验证在\(B(\infty,2)\)上\(\text{Log}\left(\frac{z-1}{z-2}\right)\)可取出单值全纯分支, 只需考虑主支. \[\begin{aligned}
      \log\left(\frac{z-1}{z-2}\right)&=\log\left(1-\frac1z\right)-\log\left(1-\frac2z\right)\\
      &=-\sum_{n=1}^\infty \frac{1}{nz^n}+\sum_{n=1}^\infty\frac{2^n}{nz^n}\\
      &=\sum_{n=1}^\infty \frac{2^n-1}{n}z^{-n}.
    \end{aligned}\]
    故\[\text{Log}\left(\frac{z-1}{z-2}\right)=\sum_{n=1}^\infty \frac{2^n-1}{n}z^{-n}+2k\pi i,\] 其中\(k\in\mathbb{Z}\).
    \item [(5)] 由于\(|\frac{5}{z}|<1\), \begin{align*}
      \frac{1}{(z-5)^n}&=\frac{1}{z^n}\frac{1}{(1-\frac{5}{z})^n}\\
      &=\frac{1}{z^n}\sum_{k=0}^\infty \frac{1}{k!}n(n+1)\cdots(n+k-1)\left(\frac{5}{z}\right)^k\\
      &=\sum_{k=0}^\infty \binom{n+k-1}{n-1}5^kz^{-n-k}. \tag*{\(\qed\)}
    \end{align*}
  \end{itemize}
  \renewcommand{\qedsymbol}{}
\end{solution}

\begin{remark}
  求Laurent级数时, 须注意在何处展开.
\end{remark}

\begin{problem}[5.2.2]
  求函数\(f(z)\)的奇点并判断其类型.
  \begin{itemize}
    \item [(3)] \(\sin\frac{1}{z-1}\).
    \item [(7)] \(\sin\left(\frac{1}{\cos\frac{1}{z}}\right)\).
    \item [(8)] \(e^{\tan z}\).
  \end{itemize}
\end{problem}

\begin{solution}
  \begin{itemize}
    \item [(3)] 可能的奇点为\(1,\infty\). 因为\(\lim\limits_{z\to1}f(z)\)不存在, \(\lim\limits_{z\to0}f\left(\frac{1}{z}\right)=\lim\limits_{z\to0}\sin\frac{z}{z-1}=0\), 所以\(1\)是本性奇点, \(\infty\)是可去奇点.
    \item [(7)] 可能的奇点为\(0,\infty,\frac{2}{(2k+1)\pi}\), 其中\(k\in\mathbb Z\). 因为\(\lim\limits_{z\to\frac{2}{(2k+1)\pi}}f(z)\)不存在, 所以\(\frac{2}{(2k+1)\pi}\) (\(k\in\mathbb Z\))是本性奇点. 从而\(0\)是非孤立奇点. 因为\(\lim\limits_{z\to 0}f\left(\frac{1}{z}\right)=\lim\limits_{z\to 0}\sin\left(\frac{1}{\cos z}\right)=\sin 1\), 所以\(\infty\)是可去奇点.
    \item [(8)] 可能的奇点为\(\infty, k\pi+\frac\pi2\), 其中\(k\in\mathbb Z\). 因为\(\lim\limits_{z\to k\pi+\frac\pi2}f(z)\)不存在, 所以\(k\pi+\frac\pi2\) (\(k\in\mathbb Z\))是本性奇点. 从而\(\infty\)是非孤立奇点. \qedhere
  \end{itemize}
\end{solution}

\begin{remark}
  须讨论\(\infty\)的类型. 注意非孤立奇点的概念.
\end{remark}

\begin{problem}[5.3.1]
  求所有\(\mathbb C\)上亚纯函数\(f\), 使得\(|f(z)|=1\)对任意\(z\in\partial B(0,1)\)成立.
\end{problem}

\begin{solution}
  设\(f\)
\end{solution}

\end{document}